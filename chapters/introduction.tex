\chapter{绪论}
\section{课题背景及研究意义}
几段话。
\section{国内外研究现状}
\subsection{全监督语义分割算法}
\subsection{弱监督语义分割算法}
\section{研究内容及结构安排}
\section{目录}
目录是学位论文的提纲,是论文各组成部分的小标题,应分别依次列出并注明页码。各级标题分别以第一章、1.1、1.1.1等数字依次标出,目录中最多列出三级标题,正文中如果确需四级标题,用(1)、(2)形式标出。学位论文的前置部分(摘要、插图索引、表格索引、符号对照表、缩略语对照表)和学位论文的主体部分(正文、参考文献、致谢、作者简介)都要在目录中列出。
\par
目录标题字体为黑体,字号为三号,居中排列,行距为固定值20磅,段落间距为段前24磅,段后18磅;目录内容中一级标题字体为黑体,字号为小四号,其余标题字体为宋体,字号为小四号。
\section{正文}
正文是学位论文的主体和核心部分。正文的一级标题居中排列,字体为黑体,字号为三号,行距为固定值20磅,段落间距为段前24磅,段后18 磅;二级标题不缩进,字体为宋体加粗,字号为小三号,行距为固定值20磅,段落间距为段前18磅,段后12磅;三级标题缩进2 字符,字体为宋体,字号为四号加粗,行距为固定值20磅,段落间距为段前12 磅,段后6磅;正文内容字体为宋体,字号为小四号,行距为固定值20 磅,段落间距为段前0磅,段后0磅。正文一般包括以下几个方面:
\subsection{绪论}
绪论是学位论文主体部分的开端,切忌与摘要雷同或成为摘要的注解。绪论除了要说明论文的研究目的、研究方法和研究结果外,还应评述与论文研究内容相关的国内外研究现状和相关领域中已有的研究成果;其次还要介绍本项研究工作的前提和任务、理论依据、实验基础、涉及范围、预期结果以及该论文在已有基础上所要解决的问题。
\subsection{各章节}
各章节一般由标题、文字叙述、图、表、公式等构成,章节内容总体要求立论正确,逻辑清晰,数据可靠,层次分明,文字通畅,编排规范。论文中若有与指导教师或他人共同研究的成果,必须明确标注;如果引用他人的结论,必须明确注明出处,并与参考文献保持一致。
\par
(1)图:包括曲线图、示意图、流程图、框图等。图序号一律用阿拉伯数字分章依序编码,如:图1.3、图2.11。 每一个图应有简短确切的图名,连同图序号置于图的正下方。图名称、图中的内容字号为五号,中文字体为宋体,英文字体为Times New Roman,行距一般为单倍行距。图中坐标上标注的符号和缩略词必须与正文保持一致。引用图应在图题右上角标出文献来源;曲线图的纵横坐标必须标注“量、标准规定符号、单位”,这三者只有在不必要标明(如无量纲等)的情况下方可省略。
\par
(2)表:包括分类项目和数据,一般要求分类项目由左至右横排,数据从上到下竖列。分类项目横排中必须标明符号或单位,竖列的数据栏中不要出现“同上”、“同左”等词语,一律要填写具体的数字或文字。表序号一律用阿拉伯数字分章依序编码,如:表2.5、表10.3。每一个表格应有简短确切的题名,连同表序号置于表的正上方。表名称、表中的内容字号为五号,中文字体为宋体,英文字体为Times New Roman,行距一般与正文保持一致。
\par
(3)公式:正文中的公式、算式、方程式等必须编排序号,序号一律用阿拉伯数字分章依序编码,如:(3-32)、 (6-21)。对于较长的公式,另起行居中横排,只可在符号处(如:+、-、*、/、$<$$>$等)转行。公式序号标注于该式所在行(当有续行时,应标注于最后一行)的最右边。连续性的公式在“=”处排列整齐。大于999的整数或多于三位的小数,一律用半个阿拉伯数字符的小间隔分开;小于1的数应将0置于小数点之前。公式的行距一般为单倍行距。
\par
(4)计量单位:学位论文中出现的计量单位一律采用国务院1984年2月27日发布的《中华人民共和国法定计量单位》标准。
\section{作者简介}
攻读博士/硕士学位期间的研究成果是指本人攻读博士/硕士学位期间发表(或录用)的学术论文,申请(授权)专利、参与的科研项目及科研获奖等情况,分别按时间顺序列出。其中,发表论文、申请(授权)专利、科研获奖只列出作者排名前3名的,参与的科研项目按重要程度最多列出5项。作者简介标题字体为黑体,字号为三号,居中排列,行距为固定值20 磅,段落间距为段前24磅,段后18磅。作者简介的正文内容严格按照本模板中的范例书写。
\section{其他}
学位论文中如果需要注释,可作为脚注在页下分别著录,切忌在文中注释;如果有附录部分,可编写在正文之后,与正文连续编页码,每一附录均另页起,附录依次用大写英文字母A、B、C……编序号,如:附录A、附录B等。
