\chapter{绪论}
% 研究目的、研究方法、研究结果
% 与论文研究内容相关的国内外研究现状和相关领域中已有的研究成果
% 本项研究工作的前提和任务、理论依据、实验基础、涉及范围、预期结果以及该论文在已有基础上所要解决的问题
\section{课题背景及研究意义}
% 语义分割的重要性
计算机视觉作为深度学习技术落地到现实生活中的重要应用领域之一,正受到越来越多研究人员和机构的关注。在深度学习发展初期,人们主要关注于如何区分一幅图像所表示的物体类别(即图像分类),或是如何分类并定位出处于同一幅图像内的多个目标(即目标识别)。而随着图像数据量的大规模增长和应用复杂性的不断提高,图像分类或目标识别算法所能输出的信息量已越来越不能满足场景理解任务的需要。人们要求计算机不仅能够从高层次上理解一幅图像表示了什么物体,还要能够站在低层次的角度理解每个像素分别对应了什么类别,这给计算机视觉研究提出了新的挑战,也推动了许多新型神经网络结构的诞生。
\par
% 语义分割的概念
得益于硬件计算能力的飞速提高,对图像的像素级分析已经成为可能,基于深度学习技术的语义分割算法也应运而生。通俗来说,语义分割指的是在给定一幅图像的条件下,为每个像素点赋予一个类别标签,使得属于同一个类别的像素聚类为一个独立的语义实体的过程。作为图像分割的一个子领域,我们可以从数学的角度将语义分割视为一类边缘分割的图像处理技术,也可以从统计学的角度将其视为一种聚类方法。但相比基于 Sobel 算子等传统图像处理方法的边缘检测而言,后者仅仅利用了一阶导数或二阶梯度等数学信息对像素值发生跃迁或渐变的区域进行识别,而前者在此基础上还对像素之间的信息关联进行了上下文建模,使得分割结果呈现语义化的特征,从而能够适应于现实语境下的应用场景。
\par
% 语义分割的应用
photoshop
% 弱监督语义分割
计算机视觉任务中的各种标注类型,弱监督语义分割

弱监督语义分割是计算机视觉中的一个重要研究领域,它旨在仅使用有限的监督信息对图像进行语义分割。
\section{国内外研究现状}
\subsection{全监督语义分割的研究现状}
一段文字。
\subsection{图像级标签下弱监督语义分割的研究现状}
一段文字。
\subsection{点线框标签下弱监督语义分割的研究现状}
一段文字。
\section{研究内容及结构安排}
本文分为三个部分。第一部分,;第二部分,;第三部分,。共分为五个章节,每个章节的具体内容如下:
\par
第一章,
\par
第二章,
\par
第三章,
\par
第四章,
\par
第五章,
