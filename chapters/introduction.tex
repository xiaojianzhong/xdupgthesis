\chapter{绪论}
% 研究目的、研究方法、研究结果
% 与论文研究内容相关的国内外研究现状和相关领域中已有的研究成果
% 本项研究工作的前提和任务、理论依据、实验基础、涉及范围、预期结果以及该论文在已有基础上所要解决的问题
\section{课题背景及研究意义}
% 语义分割的重要性
计算机视觉作为深度学习技术落地到现实生活中的重要应用领域之一,正受到越来越多研究人员和机构的关注。
在深度学习发展初期,人们主要关注于如何区分一幅图像所表示的物体类别(即图像分类),或是如何分类并定位出处于同一幅图像内的多个目标(即目标识别)。
而随着图像数据量的大规模增长和应用复杂性的不断提高,图像分类或目标识别算法所能输出的信息量已越来越不能满足场景理解任务的需要。
人们要求计算机不仅能够从高层次上理解一幅图像表示了什么物体,还要能够站在低层次的角度理解每个像素分别对应了什么类别,这给计算机视觉研究提出了新的挑战,也推动了许多新型神经网络结构的诞生。
\par
% 语义分割的概念
得益于硬件计算能力的飞速提高,对图像的像素级分析已经成为可能,基于深度学习技术的语义分割算法也应运而生。
通俗来说,语义分割指的是在给定一幅图像的条件下,为每个像素点赋予一个类别标签,使得属于同一个类别的像素聚类为一个独立的语义实体的过程。
作为图像分割的一个子领域,我们可以从数学的角度将语义分割视为一类边缘分割的图像处理技术,也可以从统计学的角度将其视为一种聚类方法。
但相比基于 Sobel 算子等传统图像处理方法的边缘检测而言,后者仅仅利用了一阶导数或二阶梯度等数学信息对像素值发生跃迁或渐变的区域进行识别,而前者在此基础上还对像素之间的信息关联进行了上下文建模,使得分割结果呈现语义化的特征,从而能够适应于现实语境下的应用场景。
\par
% 语义分割的应用
photoshop
% 弱监督语义分割
计算机视觉任务中的各种标注类型,弱监督语义分割

弱监督语义分割是计算机视觉中的一个重要研究领域,它旨在仅使用有限的监督信息对图像进行语义分割。
\section{国内外研究现状}
\subsection{全监督学习下的语义分割}
早期研究者们主要基于数学方法解决语义分割问题,这些方法对像素之间的数值关系进行建模,而不涉及语义方面的高层信息。
早在2010年,Peng等[1]应用离散数学中的图论知识,将图像视为一个图,像素视为图中的节点,通过节点之间边的权重来表示不同像素之间的相似程度,将语义分割转化为图的切割问题。
2011年,Krähenbühl等[2]基于概率论知识将条件随机场(Conditional Random Field,CRF)应用到语义分割上,在全连接的CRF模型中使用了一种高效的近似算法,为每个像素赋值一个一元损失,为每一对像素赋值一个二元损失,从而将语义分割问题转化为损失最优化的问题。
\par
2014年,全卷积网络(Fully Convolutional Network,FCN)[3]横空出世,在不引入CRF等其他技术的前提下构造了首个端到端的语义分割卷积神经网络,使用卷积层来代替传统分类任务中的全连接层,使得网络可以输出一个矩阵,在推理速度和分割精度上都取得了当时的最优效果。
自此,基于深度学习尤其是卷积神经网络的方法大放异彩,占据了语义分割领域的绝对主导地位。
\par
2015年,Noh等人[4]提出基于编码-解码结构的DeconvNet,将FCN中单独的上采样模块替换为一个完整的转置卷积网络,渐进式地恢复高层特征图的分辨率。
同年,生物医学领域的Ronneberger等人[5]提出U-Net,在编码器和解码器的对应阶段中间引入跳跃连接结构,从而减少了解码器网络在上采样过程中的信息损耗,能够在数据量极少的前提下保持细胞分割结果的高精确度。
Badrinarayanan等人[6]提出基于池化下标共享的SegNet,显著降低了模型的内存占用量。
针对CRF训练、推理速度慢的问题,Zheng等人[7]在2015年构造了一个形式等价的循环神经网络CRF-RNN,使得CRF可以被端到端训练。
考虑到空洞卷积算子在扩大卷积核感受野方面的优势,谷歌团队[8]在2017年构造了一类结合了空洞卷积和全连接CRF的DeepLab v3,在多个并行空洞卷积层的基础上添加了全局池化层来引入更丰富的上下文信息。
\subsection{弱监督学习下基于图像级标签的语义分割}
弱监督学习指的是使用监督级别较低的标签来指导模型的训练,弱监督场景下的语义分割任务涉及多种标注形式,其中类别标签仅告诉模型某一类别的出现与否,而完全不指出物体的位置或形状,具有最低的标注成本,在现代弱监督语义分割算法中应用最为广泛。
同时,由于此类标签所提供的信息量最少,因此在众多弱监督方法中也最具挑战性。
早在全监督的FCN被应用到语义分割任务之前,Pinheiro等人[9]就考虑仅使用类别标签训练分割网络,构建了一种基于多实例学习的弱监督分割框架MIL-seg,使用Log-Sum-Exp聚合层将三维特征图映射为一维向量,用分类网络的损失指导分割任务。
在FCN提出后的一个月,其原班作者借鉴MIL-seg多实例学习的思想提出MIL-FCN[10],该网络本质上等价于MIL-seg使用全局最大池化作为聚合层的特殊情况。
\par
由于类别标签不提供任何定位信息这一特殊性,基于此种标签的弱监督语义分割方法通常需要引入第三方辅助信息,如额外的训练数据或模型等。
Wei等人[11]在2015年提出STC框架,采用类似课程学习的策略将数据集按照难度进行切分,以“从简单到困难”的顺序喂给不同阶段的神经网络进行训练,从而逐步增强网络性能。
STC引入了来自互联网的额外数据样本,并使用预训练好的显著性图(saliency map)网络提供每一阶段的伪标签。
2016年,Kolesnikov等[12]将优化过程拆分为种子点损失、扩张损失和平滑损失三个部分,并且首次引入类激活图(Class Activation Map,CAM)作为伪标签。
同年,PSPNet的作者Qi等[13]提出双网络交替优化的思想,使用定位网络生成的边界框来提高分割任务精度,使用分割网络对每个像素点的分类归属来区分定位任务的正负例,达到了当时弱监督下的最优精度。
\par
STC首次证明类激活图在不需要额外辅助的前提下就能从分类网络中提取位置信息,是弱监督语义分割中伪标签的优质候选来源。
然而,由于分类网络的固有特性,类激活图的激活结果往往只包含目标最显著的像素区域,而不能完全覆盖整个目标范围,不利于分割网络进一步提高精度,因而此后众多方法都针对伪标签的质量进行改进。
2018年Ahn等[14]首次考虑到同一图像中不同像素间的关系,额外构造了一个像素相似度预测网络AffinityNet,其预测结果作为随机游走的概率参考,首次突破了PASCAL VOC数据集60.0的mIoU。
考虑到空洞卷积在DeepLab v3中的有效性,Wei等人[15]提出MDC,在分类网络的最后阶段加入类似的并行空洞卷积结构,利用不同的扩张率将类激活图扩散到目标的不同区域。
Huang等[16]以类激活图为初始点,使用传统图像处理中的种子区域生长算法实现区域扩张。
类比Dropout随机关闭节点的机制,Lee等[17]在2019年提出FickleNet,对特征图进行多次随机遮挡,以集成的方式扩大类激活图的覆盖面积。
同年,Jiang等[18]提出将训练过程中得到的类激活图进行累积。
对比学习的常规做法是将增强前后的样本使用对比损失进行距离衡量,Wang等[19]在2020年提出一种与自监督对比学习等价的弱监督语义分割网络SEAM,将仿射变换前后的图像样本同时输入到孪生网络中。
Chang等[20]引入细粒度分类的思想,在原始分类器的基础上加入一个额外的子类分类器,使得生成的类激活图更加精细化。
Zhang等[21]提出首次引入因果干预的CONTA模型,以因果图的方式切断上下文先验和类别之间的混淆关联。
考虑到Transformer在挖掘全局上下文方面的优越性,Wu等[22]在2021年提出EDAM,首次引入自注意力来挖掘图像间或图像内的像素关联性。
同年,Xu等[23]提出基于多任务学习的AuxSegNet,加入多标签分类与显著性图检测这两项辅助任务进行联合训练。
\par
对抗学习是机器学习领域非常重要的一类方法,如在安全攻防领域的对抗样本攻击,在图像生成领域的生成对抗网络等。
2017年,STC的作者Wei等人[24]首次在弱监督分割中提出对抗擦除(adversarial erasing)思想,将挖掘伪标签建模为一个迭代式的过程,在每轮迭代中擦除图像中上一轮得到的区域,迫使分类网络寻找新的类别判定依据,从而完善伪标签整体的质量。
5个月后,Kim等[25]提出两阶段的TPL,认为对抗擦除不仅适用于图像,也能在特征图上进行。
2018年,Li等[26]构建了单阶段端到端的GAIN,仅使用一轮迭代就能完成擦除过程。
Hou等[27]认为擦除前后的特征提取层可以共享,构建了由一个特征提取层和两个分类器组成的SeeNet。
2021年,Kweon等[28]将擦除过程限定在类别层面,每次擦除都仅针对图像中出现过的某一类别进行。
\subsection{弱监督学习下基于点线框标签的语义分割}
除开应用最广泛的类别标签,弱监督语义分割还可以使用包括像素点、涂鸦线、边界框在内的多种标注形式指导训练。
平均来看,每个类别标签仅需花费1秒的人工成本,相比之下,点标签需要2.4秒,线标签需要17秒,框标签需要10秒,而强监督的像素级标签需要78秒。
强弱监督的标注成本具有数量级的差距,印证了弱监督语义分割研究的必要性。
\par
像素点标签给出目标在图像中某一个随机位置的坐标,于2015年由ImageNet作者Li等人[29]引入弱监督语义分割领域。
2018年,Qian等[30]基于度量学习定义了距离损失,认为不同图像中来自同一类别的像素点属于正例对,来自不同类别的像素点属于负例对。
涂鸦线标签在目标内部画一条形状随机的曲线段,具有友好的交互特性,适合于没有明确形状的类别。
2016年,Lin等[31]构建了基于图模型的ScribbleSup,首先将图像划分为超像素簇,然后基于涂鸦线的标注情况定义图模型中的二元项。
2018年,考虑到部分交叉熵损失不足以充分利用线标签的监督信息,Tang等[32]提出Normalized Cut Loss,引入Normalized Cut这一图割方法扩充标签。
边界框标签使用矩形框描述目标的大致位置,相较之下能提供更精确的边界信息,适合于形状规则、边缘平行的类别物体。
2015年,Dai等[33]提出的BoxSup以交替迭代的方式同时实现伪标签质量和分割网络性能的提升。
2016年,Khoreva等[34]在伪标签的生成阶段加入约束条件,通过减少噪声来提升伪标签质量。
2021年,Lee等人[35]提出BBAM,加入额外的目标检测网络来提供像素级特征。
同年,Zhang等[36]提出的A2GNN创新性地通过图神经网络将输入图像转化为无向图,以图节点为单位进行优化。
\section{研究内容及结构安排}
本文分为三个部分。第一部分,;第二部分,;第三部分,。共分为五个章节,每个章节的具体内容如下:
\par
第一章,
\par
第二章,
\par
第三章,
\par
第四章,
\par
第五章,
