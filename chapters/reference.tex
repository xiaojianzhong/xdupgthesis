\chapter{参考文献示例}
这是一个英文article类型文献\cite{ChangHTD19}。
\par
这是一个中文article类型文献\cite{WangZSS21}。
\par
这是一个英文article类型文献\cite{GongL21}用于展示DOI号码的使用。
\par
这是一个英文会议类型文献\cite{TsaiCLQLB13}。
\par
这是一个网页类型文献\cite{Collinson21}用于展示misc类型条目的用法。

参考文献是文中引用的有具体文字来源的文献集合,博士学位论文参考文献一般不少于80篇,其中近5年的参考文献不少于20篇,硕士学位论文参考文献一般不少于30篇,其中近5年的参考文献不少于5篇。参考文献标题字体为黑体,字号为三号,居中排列,段落间距为段前24磅,段后18磅;参考文献若是中文文献,字体为宋体,字号为五号,若是英文文献,字体为Times New Roman,字号为五号。学位论文的撰写要本着严谨求实的科学态度,凡有引用他人成果之处,引用处右上角用方括号标注阿拉伯数字编排的序号(必须与参考文献一致),同时所有引用的文献必须用全称,不能缩写,并按论文中所引用的顺序列于文末。引用文献的作者不超过3位时全部列出,超过时列前3位,后加“等”字或“et al.”。 参考文献的著录要符合《文后参考文献著录规则》(GB/T7714-2005)要求:
\par
(1)期刊(报纸)参考文献:[序号] 主要责任者. 文献名称[文献类别代码]. 期刊(报纸)名, 年份, 卷(期): 引文页码.
\par
(2)专著参考文献:[序号] 主要责任者. 专著名称[文献类别代码]. 其他责任者. 出版地: 出版单位, 出版年份.
\par
(3)专利参考文献:[序号] 主要责任者. 专利名称: 国别, 专利号[文献类别代码]. 出版日期.
\par
(4)技术标准参考文献:[序号] 起草责任者. 标准代号-标准顺序号-发布年. 标准名称[文献类别代码]. 出版地: 出版单位,出版年份.
\par
(5)电子参考文献:[序号] 主要责任者. 题名[文献类别代码]. 获取和访问路径. [引用日期].
\par
(6)会议论文集参考文献:[序号] 编者. 论文集名. (供选择项:会议名, 会址, 开会年)出版地: 出版者, 出版年份.
\par
(7)学位论文参考文献:[序号]  主要责任者. 文献题名[文献类别代码]. 保存地: 保存单位, 年份.
\par
(8)国际、国家标准参考文献:[序号] 标准代号. 标准名称[文献类别代码]. 出版地: 出版者, 出版年.
\par
(9)报告类参考文献:[序号] 主要责任者. 文献题名[文献类别代码]. 报告地: 报告会主办单位, 年份.
\par
参考文献著录中的文献类别代码:
\par
(1)普通图书:M
\par
(2)会议录:C
\par
(3)汇编:G
\par
(4)报纸:N
\par
(5)期刊:J
\par
(6)学位论文:D
\par
(7)报告:R
\par
(8)标准:S
\par
(9)专利:P
\par
(10)数据库:DB
\par
(11)计算机程序:CP
\par
(12)电子公告:EB
\par
载体类型:
\par
网络:OL
\par
磁带:MT
\par
磁盘:MK
\par
光盘:CD